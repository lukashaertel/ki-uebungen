\begin{enumerate}
	\item Ein Agent, vollständig beobachtbar, diskret, episodisch
	\begin{enumerate}
		\item Sortierapparat: die Umgebung ist vollständig im zu sortierenden Gegenstand. Es gibt eine diskrete anzahl an Kriterien und damit Aktionen, z.B. Farben. Die Sortierung wirkt sich nicht auf andere Sortierungen aus, also episodisch.
		
		Das Szenario ist nicht deterministisch, da nach einer Sortierung die nächste Aktion nicht fest steht.
		
		\item Drehkreuz in U-Bahn, Welt wird vollständig durch den Kartenleser beobachtet. Die Aktionen sind durchlassen oder nicht durchlassen, damit diskret. Die Aktion wirkt sich nicht auf folgende Passagiere auf.
	\end{enumerate}
	
	\item Ein Agent, kontinuierlich, partiell beobachtbar, dynamisch
	\begin{enumerate}
		\item 
		
		\item 
	\end{enumerate}
	
	\item Mehrere Agenten, dynamisch, deterministisch, kontinuierlich
	\begin{enumerate}
		\item 
		
		\item 
	\end{enumerate}
	
	\item Mehrere Agenten, indeterministisch, diskret
	\begin{enumerate}
		\item Spiel mit KI-Gegner in Bomberman: Spieler agiert nichtdeterministisch, aber diskret, da man nur auf Feldern stehen kann. Es spielen mehrere Agenten als Wettbewerber mit.
		
		\item High Fxequency Trading: jeder HFT Akteur handelt diskrete Beträge eines Wertpapiers. Die Werte ändern sich aber auch durch Einflüsse aus der echten Welt und durch Handel auf Börsenparkett, etc.
	\end{enumerate}
	
\end{enumerate}
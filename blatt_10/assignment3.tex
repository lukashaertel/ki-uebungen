\begin{enumerate}[label=\alph*)]
	\item Ein Agent, vollständig beobachtbar, diskret, episodisch
	\begin{itemize}
		\item Sortierapparat: die Umgebung ist vollständig im zu sortierenden Gegenstand. Es gibt eine diskrete anzahl an Kriterien und damit Aktionen, z.B. Farben. Die Sortierung wirkt sich nicht auf andere Sortierungen aus, also episodisch.
		
		Das Szenario ist nicht deterministisch, da nach einer Sortierung die nächste Aktion nicht fest steht.
		
		\item Drehkreuz in U-Bahn, Welt wird vollständig durch den Kartenleser beobachtet. Die Aktionen sind durchlassen oder nicht durchlassen, damit diskret. Die Aktion wirkt sich nicht auf folgende Passagiere auf.
	\end{itemize}
	
	\item Ein Agent, kontinuierlich, partiell beobachtbar, dynamisch
	\begin{itemize}
		\item Cruise-control: Gas-geben und Bremsen ist für eine ruhige Fahrt kontinuierlich (langsame vs. plötzliche Beschleunigung). Es kann aber nicht beobachtet, ob der Wagen einen Berg hinauf/hinunter fährt. Wagen vor dem eigenen können die Welt dynamisch verändern und Bremsung erfordern.
		
		\item Luftabwehrrakete: Zielansteuerung ist kontinuierlich um das Projektil nicht durch abrupte Manöver zu zerstören, die Zielerfassung hat eine feste Auflösung, die das Ziel mit abnehmender Entfernung genauer wahrnimmt, also partiell beobachtbar. Das Ziel kann ausweichen oder das Projektil täuschen.
	\end{itemize}
	
	\item Mehrere Agenten, dynamisch, deterministisch, kontinuierlich
	\begin{itemize}
		\item 
		
		\item 
	\end{itemize}
	
	\item Mehrere Agenten, indeterministisch, diskret
	\begin{itemize}
		\item Spiel mit KI-Gegner in Bomberman: Spieler agiert nichtdeterministisch, aber diskret, da man nur auf Feldern stehen kann. Es spielen mehrere Agenten als Wettbewerber mit.
		
		\item High Fxequency Trading: jeder HFT Akteur handelt diskrete Beträge eines Wertpapiers. Die Werte ändern sich aber auch durch Einflüsse aus der echten Welt und durch Handel auf Börsenparkett, etc.
	\end{itemize}
\end{enumerate}

\paragraph{Allgemeines Wissen} Aus dem Übungstext übernommen. Erst werden Tageszeiten, Aufgaben und ihre Dauer, Voraussetzungen und bekannte Startzeiten angegeben.

\lstinputlisting[ %
	language=Prolog, %
	firstnumber=1, %
	firstline=1, %
	lastline=15, %
	caption={problem.lp}] %
{assignment3/problem.lp}


\paragraph{Abgeleitetes Wissen} Berechnung der Endzeit einer Aufgabe, Constraints für Aufgabenvoraussetzung, Überprüfung auf Tageszeit, Vermeidung von Überlappung, sowie Generator für alle möglichen Startzeiten.

Überlappungen werden durch die belegten Stunden abgefangen, wie durch \lstinline+occupied/2+ vorbereitet und das letzte Constraint angegeben.

\lstinputlisting[ %
	language=Prolog, %
	firstnumber=16, %
	firstline=16, %
	lastline=39, %
	caption={problem.lp (contd.)}]%
{assignment3/problem.lp}

\paragraph{Resultat} Nach Berechnung über \lstinline+gringo problem.lp | clasp > result.txt+ ergeben sich die folgenden Belegungen.

\begin{enumerate}
	\item[\footnotesize 9 Uhr] Geldautomat
	\item[\footnotesize 10 Uhr] KI-Vorlesung
	\item[\footnotesize 12 Uhr] Einkaufen
	\item[\footnotesize 14 Uhr] Grillfeier
	\item[\footnotesize 17 Uhr] KI-Übung
\end{enumerate}

\lstinputlisting[caption={result.txt}] %
{assignment3/result.txt}